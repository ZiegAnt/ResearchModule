\subsection{Data management}\label{section:Data_management}

\paragraph*{Software}
Data processing, inversions and modelling will be performed in open-source packages in Python (e.g. PyGIMLI, Numpy etc.). It is possible that SeismicUnix will be used for seismic data inspection.

\paragraph*{Inputs}
Input data for the Master thesis research will be previously acquired field data. A summary of the different data is shown in table \ref{table:Input_data}. As the data was acquired as part of previous thesis works at the Ruhruniversität Bochum and not published yet, thos files will be stored in a final repository which will be made available for the supervisors. This will possibly be done via Sciebo. This will be done as well with additional data that might be acquired as part of this thesis work.

\begin{table}[!htb]
\caption{Overview of input data.}
\centering
\begin{tabular}{lcc}
%{m{15mm} m{70mm} m{18mm}}
\hline 
\textbf{Method} & \textbf{Year} & \textbf{Size}  \\ \hline 
Magnetics & 2011 & 1.6MB\\
& 2012 & 212KB\\
ERT & 2015 & 158MB\\
& 2020 & 128KB\\
Seismics & 2015 & 683MB\\
& 2017 & 656MB\\
Photogrammetry & 2020 & 12GB\\
& & \textbf{Total:} 14GB\\
 \hline 
\end{tabular}
\label{table:Input_data}
\end{table}

\paragraph*{Outputs}outputs: Formats, data size, where stored
Intermediate and final results of the thesis work include processed data, inversion results in form of models, Python scripts and Figures generated to visualize results. Those will be stored in a Github repository which will be made available for the supervisors. For the time of the research this repository will be kept private. However, after finishing the thesis work the repository might be made public depending on the outcome of the thesis work. Also, the report in form of .tex and other \LaTeX-formats will be stored in the repository to ensure version control of the writing process. It also makes an easy access and commenting for the supervisors possible.
