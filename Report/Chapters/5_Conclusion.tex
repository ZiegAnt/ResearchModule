\section{Conclusion}\label{section:Conclusion}

Aim of this project is to investigate how different geophysical methods are sensing a diatreme structure in a layered subsurface. The 2D synthetic data study showed, that electrical resistivity and traveltime tomography can image the diatreme, especially the upper boundary. Of all methods the Dipole-Dipole configuration returned the best image but as the method is prone to noise different more robust configurations and methods should be considered as well. Furthermore, section \ref{section:Discussion} discussed the lack of resolution close to the diatreme boundaries in the results of the traveltime tomography. 

The upcoming Master thesis project will be based on previously acquired field data including the methods of seismic refraction, seismic reflection, magnetics and ERT. As data quality might cause difficulties during the joint inversion process or regions with data sparsity might be identified during the project, additional field data could be acquired to improve the modelling of the diatreme structure. The synthetic data study of this project suggests that ERT data is most suitable to recover information about the lateral boundaries of the diatreme as well as the shallow subsurface including the upper boundary of the diatreme. An ERT survey with the Dipole-Dipole electrode configuration should be complemented by other more robust configurations as the signal to noise ratio might cause troubles. For improving the imaging of deeper parts seismic refraction surveys with long receiver spreads and a stronger source like a shotgun needs to be considered. Also gravity data could be acquired to gain more information about deeper structures as the density contrast should result in an observable anomaly.

This work also stressed the importance of the regularization during the data inversion, namely the smoothing constraint. Here, the final inversion parameters were found using semi-random testing and the decision was purely based on the data misfit measure $\chi$. Since the goal of the Master thesis research is the generation of an optimized data-driven model, further fine-tuning of the inversion parameters and more sophisticated criteria for the inversion constraint are necessary as it will lead to improvements of the inversion results and the resulting image of the volcanic diatreme structure.