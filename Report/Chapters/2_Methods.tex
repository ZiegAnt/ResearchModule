\section{Methods}\label{section:Methods}

\subsection{Mesh generation}\label{section:Mesh}
+ why unstructired mesh \\
+ what method is used for meshing\\
+ what and why values of different regions\\

To discretize the 2D subsurface geometry, especially the diatreme structure, as good as possible an unstructured, triangualr mesh is used. This is done in PyGIMLI by using the funtion \textit{pygimli.meshtools.createMesh()} which is calling the two-dimensional quality mesh generator and delaunay triangulator \textit{Triangle} \citep{shewchuk1996triangle}. The resulting mesh is shown inf Figure \ref{figure:mesh} and consists mainly of triangles without small and large angles and is therefore suitable for finite element calculations. 

Different material properties, i.e. seismic P-wave velocity, electrical resistivity and density, are assigned to the  different geologic formations that are indicated in Figure \ref{figure:synthetic_model}. The values are based on previous geophysical studies of the area as for example \citet{NiklasPlumpe.2015,TimGilberti.2020} and literature as ....

\subsection{Electrical Resistivity Tomography}\label{section:ERT}
+ few sentences on theory
+ what methods are used
+ what is used for inversion

\subsection{Traveltime Toomgraphy}\label{section:TT}
+ few sentences on theory
+ what methods are used
+ what inversion methods are used

\subsection{Gravimetry}\label{section:Gravimetry}
+ few sentences on theory
+ what methods are used