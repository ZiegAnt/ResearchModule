\section*{Abstract}

This paper presents a 2D synthetic imaging study of a simplified volcanic diatreme structure in a layered subsurface. It is designed to mimic a potential diatreme structure of a phonolitic eruption center as it is expected in the region of the Rockeskyller Kopf in the Westeifel, Germany. In the upcoming Master research project, acquired field data will be jointly inverted to model the volcanic eruption center in order to improve existing knowledge about the geologic history of the study area. Goal of this study is to investigate the sensitivity as well as its limitations of different geophysical methods to such a geologic structure. Therefore, it will help to understand the information content in the data sets which will be useful for the inversions as well as for additional data acquisition. The synthetic study includes electrical resistivity tomography (ERT), traveltime tomography and gravity data. Model construction, forward calculations and data inversion is performed using the open-source library pyGIMLi \citep{Ruecker2017}. The material properties of the forward model are based on previously acquired data of the region around the Rockeskyller Kopf as well as typical values found in the literature. ERT and traveltime inversion results represent a error-weighted, smoothness-constrained least-squares solution to the inverse problem and are determined using the corresponding routine in pyGIMLi. The results show a good representation of the subsurface geometry. The best results are obtained by the Dipole-Dipole ERT configuration, however, the influence of noise suggests that complementary methods or electrode configurations need to be acquired in the field. Due to the irregular diatreme boundaries as well as an unfavorable low velocity of the diatreme, the method is lacking to image the inside of the structure homogeneously which makes an accurate interpretation more difficult. The modeled gravity response of the subsurface also suggests that additional gravity data can be acquired in the field to add more information about deeper parts of the structure as the ERT and traveltime tomography methods are limited to the shallow subsurface. This study also stresses the influence of regularization parameters during data inversion and the importance of a sophisticated choice of the parameters as it has a strong influence on the final images.