\section{Discussion}\label{section:Discussion}

As seen in Figure \ref{figure:ERT_inversion_comp} all three configurations result in a similar subsurface resistivity distribution. Comparing them to the superimposed initial starting model that was used for the data generation, all ERT inversions results represent the upper soil layer as well as the upper part of the diatreme structure accurately. However, there a differences observable between the Dipole-Dipole and the remaining two configuration as the Soil layer appears to be more homogeneous for the Schlumberger and Wenner configuration when comparing it to the Dipole-Dipole result. This is resulting from the noise and the inversion settings. As the Dipole-Dipole configuration has more measurements overall more data is involved in the inversion process. In order to reach a data misfit of approximately $\chi^2=1$, the smoothing parameter $\lambda$ was comparably low with a value of 3 which implies less smoothing and a higher variance of the model parameter. This on the other hand allows the model parameters to appear in a more geologic way as a less smooth transition between soil, sandstone and diatreme can be observed. This is desirable as the boundaries of the diatreme, especially the ones on the sides, are retrieved slightly better when comparing it to the other two configurations. In this case, the inversion parameters are found by trial and error to reach a data misfit close to 1. A more sophisticated approach for determining the value of the regularization parameter, like the L-curve criterion, is not used in this project due to the limited amount of time, however, it could improve inversion results of the Schlumberger and Wenner configuration towards a slightly better fit with the initial starting model \citep{lawson1995solving}.

The soil layer can be retrieved quite accurately from the synthetic traveltime data which can be derived from Figure \ref{figure:TT_inversion_comp}. Using the superimposed synthetic subsurface geometry, the high-velocity structures that are mentioned in section \ref{section:Res_TT} can be identified as the upper sandstone layer. Due to no ray coverage at the model boundaries the high-velocity structures appear more like an elliptic object rather than a horizontal layer. The upper boundary as well as the left boundary of the diatreme structure are imaged properly, while the right side of the diatreme shows a more gradual transition which makes the interpretation of the boundary position more difficult. As already discussed for the ERT inversion result, this smooth transition could be improved by investigations of the regularization parameter $\lambda$. The traveltime inversion images the insides of the diatreme rather heterogeneous in form of two medium and one high-velocity region. Averaging the inside of the diatreme structure returns a good approximation of the true velocity, however, without the superimposed subsurface geometry an interpretation would probably result in several structures rather than combining all three anomalies in one structure. A possible reason for the lack of imaging of the inside of the diatreme is the comparably low velocity. During traveltime tomography, rays are  circumventing low-velocity regions which results in a lower data coverage and less reliable values. As the diatreme has a lower velocity than the surrounding sandstone, the sides of the diatreme are sensed less by the virtual rays and therefore could cause the lack of imaging. This reasoning is supported by the ray coverage plot on the right of Figure \ref{figure:TT_inversion_comp} as the low-velocity regions coincide with lower ray coverage. Especially close to the right boundary of the diatreme at a distance of approximately 80 m, several cells without ray coverage can be observed. 

After discussing the results of the ERT and traveltime inversion separately the methods are now compared with each other. Both methods are able to accurately recover the upper soil layer as well as the upper boundary of the diatreme. They also return accurate estimates for the resistivity and velocity of the soil and sandstone layer. The ERT method resolves the lateral boundaries of the diatreme structure slightly better and images the diatreme as a homogeneous structure which the traveltime tomography is lacking. Especially the Dipole-Dipole configuration images the diatreme boundaries accurately but is prone noise in the measurements. The traveltime tomography as well as the other two ERT configurations are more robust but perform slightly worse in the imaging of the diatreme. In the field, the traveltime tomography has better options to acquire data in a longer setup (i.e. with a longer seismic line) as the signal to noise ration can be improved by stacking or a stronger source signal \citep{kearey2002introduction}. Therefore, the penetration depth of the traveltime tomography survey can be improved such that deeper parts of the diatreme could be imaged. Extending ERT lines on the other hand might be more challenging as with increasing electrode distance the measured signal becomes very weak.  It is much harder to overcome that problem such that the limiting penetration depth is more shallow.  

In addition to the ERT and the traveltime data, synthetic gravity measurements are generated. The resulting anomaly of the diatreme structure shows a magnitude of 0.3 mGal. Commercial gravimeters that are commonly used in gravimetric surveys measure the gravitational acceleration with an accuracy of approximately 0.01 mGal, some even provide an accuracy of 0.001 mGal \citep{BGR_grav}. Therefore, this anomaly should be clearly observable in gravimeter measurements in the field. Note that in the synthetic model, topographic effects were neglected and the 2D subsurface section was assumed to be horizontally layered around the diatreme. In reality, topography, subsurface heterogeneities, 3D effects or deeper structures like small volcanic intrusions might influence the measurements and limit the detectability. However, the gravity anomaly should be still be measurable. Depending on how accurate the gravity measurement are and how strong external factors are influencing the survey, gravity data could hold information about the deep subsurface below the depth of penetration of ERT or traveltime tomography. Since the ERT method and traveltime tomography image the upper 30 m quite accurate, a combination with a potential gravity data set could improve the deeper parts of a model or even extend it.
