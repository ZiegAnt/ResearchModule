\section{Discussion}\label{section:Discussion}
+ comment on differences between ert configs
+ comment on TT result
+ comment comment  on what method resolves what parts
+ include gravimetric data 

ERT
all show overall similar resistivity distributions
all return a good soil layer and diatreme top boundary
DD upper layer appears more heterogeneous (due to Noise affecting more measurements there, more sensitive to noise cause of electrode configurations)
DD slightly deeper 
SLM and WA very similar without significant differences
DD resolves diatreme upper side part a little bit better, also top boundary slightly more accurate (less smooth, might be due to smoothing in inversion)
DD resistivities closer to initial model

TT
Soil layer retrieved good
CSS below penetration of virtual rays/ not sensed
SS only close to diatreme boundaries but fairly good velocity estimate, no coverage at boundaries 
diatreme appears heterogeneous with two low and 1 higher velocity zone, average is in right range
diatreme high velocity thing due to 
upper boundary and sides accurate, left little better

ERT vs. TT
ERT shows diatreme structure more mohogenous
both accurately recover soil layer and depth of diatreme/top boundary
similar depth 
great accordance with each other

Reason why boundaries are not as clear:
smoothing in inversion/Regularization
mesh based on data acquisition geometry and doesnt sampe boundaries exactly

Gravity data is not inverted but shows clear anomaly of 0.3mGal
accuracy of gravimeter 0.01-0.001mGal \cite{BGR_grav}
measurable

